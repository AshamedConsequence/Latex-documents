\documentclass[14pt, letterpaper]{article}
\author{Maxime Morize, Alexander Khvoshchev}
\title{Fiche de synth\`ese sur l'arithm\'etique}


\begin{document}
\maketitle
\section{D\'efinitions}
\begin{itemize}
  \item{Un entier naturel:}
    Un entier naturel est un entier positif ou non nul.
  \item{Un diviseur commun:}
    Un diviseur commun \`a deux nombres \(a\) et \(b\) divise \`a la fois \(a\)
  \item{Le PGCD:}
    Le PGCD (Le\emph{ Plus Grand Commun Diviseur}) est le plus grand diviseur
    commun de deux nombres ou plus.
  \item{Etre premier entre nombres:}
    Deux nombres sont premiers entre eux lorsque leurs seuls diviseur commun est
    1, soit le PGCD est \'egale \`a 1.
  \item{Un nombre premier:}
    Un nombre premier est un nombre naturel qui ne poss\`ede que 2 diviseurs, 1
    et lui m\^eme.
  \item{Une fraction irr\'eductible:}
    Une fraction est irr\'eductible lorsque son num\'erateur et son
    d\'enominateur sont premiers entre eux.

\end{itemize}
\section{Savoir faire}
\begin{itemize}
\item{D\'eterminer la liste des diviseurs communs \`a deux nombres entiers
    naturels \(a\) et \(b\)}
    \begin{itemize}
      \item{On d\'etermine la liste diviseurs de \(a\) et de \(b\)}
      \item{On en d\'eduit la liste des diviseurs communs}
    \end{itemize}

\item{D\'eterminer le PGCD de \(a\) et \(b\)}
    \begin{itemize}
      \item{On determine la liste des diviseurs communs \`a \(a\) et \(b\)}
      \item{On cherche le plus grand de ces diviseurs communs, qui est le PGCD}
    \end{itemize}

\item{D\'eterminer si deux nombres sont premiers entre eux}
    \begin{itemize}
      \item{On determine le PGCD de ces nombres}
      \item{On regarde si le PGCD est \'egal \`a 1}
    \end{itemize}

\item{D\'eterminer si un entier est premier en \'ecrivant la liste de ses diviseurs}
    \begin{itemize}
      \item{On \'ecrit la liste des diviseurs de ce nombre}
      \item{Si la liste est compos\'ee de 1 et du nombre lui m\^eme, c'est un
        nombre premier}
    \end{itemize}
  \item{D\'ecomposer en facteurs premier un nombre}
    \begin{itemize}
      \item{On divise par le plus petit nombre premier qui donne un r\'esulat
        entier}
        \item{On continue de faire ceci jusqu'\`a ce qu'on retrouve 1 comme r\'esultat}
    \end{itemize}

\end{itemize}
\section{Exemples}
\section{Propi\'et\'es}
\begin{itemize}
    \item{stuff}
\end{itemize}

\end{document}
